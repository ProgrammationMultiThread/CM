% SPDX-License-Identifier: CC-BY-SA-4.0
% Author: Matthieu Perrin
% Part: 
% Section: 
% Sub-section: 
% Frame: 

\begingroup

\begin{frame}{Limites de la parallélisation}

  \begin{alertblock}{Question}
    \begin{itemize}
    \item Un sondeur a besoin de 22 jours pour interroger 1000 sondés.
    \item Combien de temps faut-il à 22 sondeurs pour faire le même sondage ?
    \end{itemize}
  \end{alertblock}

  \pause

  \begin{alertblock}{Question}
    \begin{itemize}
    \item Une éléphante a besoin de 22 mois pour donner la vie.
    \item Combien de temps faut-il à 22 éléphantes pour faire un éléphanteau ?
    \end{itemize}
  \end{alertblock}

  \pause

  \begin{block}{Cas général}
    \begin{itemize}
    \item Une proportion $p$ du temps d'exécution peut être parallélisée.
    \item Une proportion $1-p$ ne le peut pas.
    \end{itemize}
  \end{block}

\end{frame}

\endgroup
\endinput
