% SPDX-License-Identifier: CC-BY-SA-4.0
% Author: Matthieu Perrin
% Part: 
% Section: 
% Sub-section: 
% Frame: 

\begingroup

\begin{frame} {Approches pour la programmation parallèle}
  \begin{exampleblock}{Programmation multi-processus}
    \begin{itemize}
    \item Plusieurs processus, un thread par processus
    \item Communication par passage de messages
    \item Avantages : 
      \begin{description}
      \item [Robustesse :] les processus sont isolés les uns des autres par l'OS
      \item [Répartition :] on peut répartir les processus sur un réseau de machines (passage à l'échelle)
      \end{description}
    \end{itemize}
  \end{exampleblock}
  
  \vfill
  \begin{exampleblock}{Programmation multi-threads}
    \begin{itemize}
    \item Un seul processus, plusieurs threads au sein du même processus
    \item Communication par mémoire partagée
    \item Avantages : 
      \begin{description}
      \item [Efficacité :] données et références partagées (pas de copie) %communication et commutation de contexte en espace utilisateur.
      \item [Simplicité :] intégration presque transparente au sein des langages de programmation %(communication par référence et par valeur)
      \end{description}
    \end{itemize}
  \end{exampleblock}
\end{frame}

\endgroup
\endinput
