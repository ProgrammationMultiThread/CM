% SPDX-License-Identifier: CC-BY-SA-4.0
% Author: Matthieu Perrin
% Part: 
% Section: 
% Sub-section: 
% Frame: 

\begingroup

\tikzset{
  inout/.style={
    draw,
    fill=structure!30,
    circle,
    text width=3.5mm,
    inner sep=0pt,
    align=center,
    font=\tiny,
  },
  procs/.style={
    draw,
    fill=alert!30,
    minimum width=10mm,
    text height=2mm,
    inner sep=3pt,
    align=center,
    font=\small,
  },
  nopro/.style={
    draw=none,
    fill=none,
    minimum width=10mm,
    text height=2mm,
    inner sep=3pt,
  },
  back/.style={
    draw,
    fill=black!40,
    fit=(p1)(p2)(p3),
    inner sep=5pt,
  },
  arrow/.style={
    -latex,
  },
}

\begin{frame}{Point vocabulaire}

  \begin{tikzpicture}[x=48mm, y=35mm]

    \node[align=center] at (2,2.6) {Une seule \\ unité d'exécution}; 
    \node[align=center] at (3,2.6) {Plusieurs \\ unité d'exécution}; 
    \node[align=center] at (1.3,2) {Problème  \\ séquentiel}; 
    \node[align=center] at (1.3,1) {Problème  \\ de \\ concurrence}; 

    \node[anchor=base] at (2,2) {
      \begin{tikzpicture}[x=15mm, y=7mm, baseline=(p2)]
        \node        (tl) at (2, 0) {Séquentiel};
        \node[inout] (i2) at (1, 2) {$I$};
        \node[nopro] (p1) at (2, 3) {};
        \node[procs] (p2) at (2, 2) {$p$};
        \node[nopro] (p3) at (2, 1) {};
        \node[inout] (o2) at (3, 2) {$O$};
        
        \path[arrow] (i2) edge (p2.west);
        \path[arrow] (p2.east) edge (o2);
        \begin{scope}[background]
          \node[back] {};
        \end{scope}
      \end{tikzpicture}
    };

    \node[anchor=base] at (3,2) {
      \begin{tikzpicture}[x=15mm, y=7mm, baseline=(p2)]
        \node        (tl) at (2, 0) {Parallèle};
        \node[inout] (i2) at (1, 2) {$I$};
        \node[procs] (p1) at (2, 3) {$p_1$};
        \node[procs] (p2) at (2, 2) {$p_2$};
        \node[procs] (p3) at (2, 1) {$p_3$};
        \node[inout] (o2) at (3, 2) {$O$};
        
        \path[arrow] (i2) edge (p1.west);
        \path[arrow] (i2) edge (p2.west);
        \path[arrow] (i2) edge (p3.west);
        
        \path[arrow] (p1.east) edge (o2);
        \path[arrow] (p2.east) edge (o2);
        \path[arrow] (p3.east) edge (o2);
        
        \begin{scope}[background]
          \node[back] {};
        \end{scope}
      \end{tikzpicture}
    };
    
    \node[anchor=base] at (2,1) {
      \begin{tikzpicture}[x=15mm, y=7mm, baseline=(p2)]
        \node        (tl) at (2, 0) {Centralisé};
        \node[inout] (i1) at (1, 3) {$I_1$};
        \node[inout] (i2) at (1, 2) {$I_2$};
        \node[inout] (i3) at (1, 1) {$I_3$};
        \node[nopro] (p1) at (2, 3) {};
        \node[procs] (p2) at (2, 2) {$p$};
        \node[nopro] (p3) at (2, 1) {};
        \node[inout] (o1) at (3, 3) {$O_1$};
        \node[inout] (o2) at (3, 2) {$O_2$};
        \node[inout] (o3) at (3, 1) {$O_3$};
        
        \path[arrow] (i1) edge (p2.west);
        \path[arrow] (i2) edge (p2.west);
        \path[arrow] (i3) edge (p2.west);
        
        \path[arrow] (p2.east) edge (o1);
        \path[arrow] (p2.east) edge (o2);
        \path[arrow] (p2.east) edge (o3);
        \begin{scope}[background]
          \node[back] {};
        \end{scope}
      \end{tikzpicture}
    }; 

    \node[anchor=base] at (3,1) {
      \begin{tikzpicture}[x=15mm, y=7mm, baseline=(p2)]
        \node        (tl) at (2, 0) {Réparti (distributed)};
        \node[inout] (i1) at (1, 3) {$I_1$};
        \node[inout] (i2) at (1, 2) {$I_2$};
        \node[inout] (i3) at (1, 1) {$I_3$};
        \node[procs] (p1) at (2, 3) {$p_1$};
        \node[procs] (p2) at (2, 2) {$p_2$};
        \node[procs] (p3) at (2, 1) {$p_3$};
        \node[inout] (o1) at (3, 3) {$O_1$};
        \node[inout] (o2) at (3, 2) {$O_2$};
        \node[inout] (o3) at (3, 1) {$O_3$};
        
        \path[arrow] (i1) edge (p1.west);
        \path[arrow] (i1) edge (p2.west);
        \path[arrow] (i1) edge (p3.west);
        \path[arrow] (i2) edge (p1.west);
        \path[arrow] (i2) edge (p2.west);
        \path[arrow] (i2) edge (p3.west);
        \path[arrow] (i3) edge (p1.west);
        \path[arrow] (i3) edge (p2.west);
        \path[arrow] (i3) edge (p3.west);
        
        \path[arrow] (p1.east) edge (o1);
        \path[arrow] (p1.east) edge (o2);
        \path[arrow] (p1.east) edge (o3);
        \path[arrow] (p2.east) edge (o1);
        \path[arrow] (p2.east) edge (o2);
        \path[arrow] (p2.east) edge (o3);
        \path[arrow] (p3.east) edge (o1);
        \path[arrow] (p3.east) edge (o2);
        \path[arrow] (p3.east) edge (o3);
        
        \begin{scope}[background]
          \node[back] {};
        \end{scope}
      \end{tikzpicture}
    };
    
  \end{tikzpicture}

\end{frame}

\endgroup
\endinput
