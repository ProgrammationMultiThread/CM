% SPDX-License-Identifier: CC-BY-SA-4.0
% Author: Matthieu Perrin
% Part: 
% Section: 
% Sub-section: 
% Frame: 

\begingroup

\begin{frame}{Parallélisme versus concurrence}

  \vFill

  \begin{block}{Principale difficulté}
    \begin{description}
    \item[Parallélisme :] Identifier les tâches indépendantes
    \item[Concurrence :] Gérer l'incertitude
    \end{description}
    \begin{shadequote}{Leslie Lamport}
      A distributed system is one in which the failure of a computer you didn't even know existed can render your own computer unusable.
    \end{shadequote}
  \end{block}

  \vFill

  \uncover<2->{
    \begin{block}{Contexte d'exécution}
      \begin{description}
      \item[Parallélisme :] Choisi en fonction du problème à résoudre
      \item[Concurrence :] Imposé \textit{a priori} par le problème et l'environnement
      \end{description}
    \end{block}
  }

  \vFill

  \uncover<3->{
    \begin{block}{Passage à l'échelle}
      \begin{minipage}{.5\textwidth}
        \begin{description}
        \item[Parallélisme :] $S_p(n) = \Omega(n)$
        \end{description}
      \end{minipage}\begin{minipage}{.45\textwidth}
        \begin{description}
        \item[Concurrence :] $S_c(n) = \Omega(1)$
        \end{description}
      \end{minipage}
    \end{block}
  }

  \vFill

  \footnoteref{M. Raynal. \textit{Parallel Computing vs. Distributed Computing: A Great Confusion? (Position Paper).} Euro-Par Workshops (2015)}
  
\end{frame}

\endgroup
\endinput
