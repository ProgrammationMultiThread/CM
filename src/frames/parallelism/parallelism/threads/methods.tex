% SPDX-License-Identifier: CC-BY-SA-4.0
% Author: Matthieu Perrin
% Part: 
% Section: 
% Sub-section: 
% Frame: 

\begingroup

\begin{frame}[fragile]{Principales méthodes de la classe gérant les threads}

  \begin{block}{Obtenir le thread qui contrôle la tâche en cours}
    \begin{description}
    \item[Java :] \lstinline|Thread.currentThread()| \hfill
    \item[C++ :] \lstinline|std::this_thread::get_id()|
    \end{description}
  \end{block}

  \pause
  \vfill

  \begin{block}{Endormir le thread courant pour \lstinline{ms} millisecondes}
    \begin{description}
    \item[Java :] \lstinline|Thread.sleep(ms)| \hfill
    \item[C++ :] \lstinline|std::this_thread::sleep_for(std::chrono::milliseconds(ms));|
    \end{description}
  \end{block}

  \pause
  \vfill

  \begin{block}{Attente de la terminaison d'un thread \lstinline{t}}
    \begin{description}
    \item[Java ou C++ :] \lstinline|t.join()|
    \item [\alert{ attention :}] en C++, un thread doit être détaché avant la destruction de l'objet $t$ (\lstinline|t.join()| ou \lstinline|t.detach()|). 
    \end{description}
  \end{block}

\end{frame}

\endgroup
\endinput
