% SPDX-License-Identifier: CC-BY-SA-4.0
% Author: Matthieu Perrin
% Part: 
% Section: 
% Sub-section: 
% Frame: 

\begingroup

\begin{frame}[fragile]
  \frametitle{Principales méthodes de la classe gérant les threads}

  \vfill
\begin{block}{Obtenir la référence sur le thread qui contrôle la tâche en cours}
   \structure{Java :} \lstinline|Thread.currentThread()| \hfill
   \structure{C++ :} \lstinline|std::this_thread|
\end{block}
\pause
  \vfill
  \vfill
\begin{block}{Endormir le thread \lstinline{t} pour \lstinline{ms} millisecondes}
   \structure{Java :} \lstinline|t.sleep(ms)| \hfill
   \structure{C++ :} \lstinline|t.sleep_for(std::chrono::milliseconds(ms))|
 \end{block}
  \vfill
  \vfill
\pause
\begin{block}{Attente de la terminaison d'un thread \lstinline{t}}
 \begin{description}
 \item[Java ou C++ :] \lstinline|t.join()|
  \item [\alert{ attention :}] en C++, un thread doit être détaché avant la terminaison du thread qui l'a lancé (\lstinline|t.detach()|). 
 \end{description}
\end{block}
  \vfill
\end{frame}

\endgroup
\endinput
