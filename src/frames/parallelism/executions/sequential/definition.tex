% SPDX-License-Identifier: CC-BY-SA-4.0
% Author: Matthieu Perrin
% Part: 
% Section: 
% Sub-section: 
% Frame: 

\begingroup

\begin{frame}[fragile]{Cohérence séquentielle}

  \begin{block}{Il faudrait que...}
    \begin{shadequote}{Leslie Lamport}
      The result of any execution is the same as if the operations of all the processors were executed in some sequential order,
      and the operations of each individual processor appear in this sequence in the order specified by its program.
    \end{shadequote}
  \end{block}

  \vfill

  \begin{block}{Entrelacement}
    Un \structure{entrelacement} d'un ordre partiel $\le$ est un ordre total qui contient $\le$. 
  \end{block}
  \begin{block}{Cohérence séquentielle}
    Une exécution est \structure{séquentiellement cohérente} si son \structure{résultat observable} est \alert{le même que} celui d'un \structure{entrelacement} de son ordre ``happened before''.
  \end{block}

  \vfill
  \begin{citing}
  \item[L79] Leslie Lamport. \textit{How to make a multiprocessor computer that correctly executes multiprocess programs.} IEEE ToC (1979)
  \end{citing}

\end{frame}

\endgroup
\endinput
