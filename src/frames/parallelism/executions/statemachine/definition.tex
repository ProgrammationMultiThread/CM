% SPDX-License-Identifier: CC-BY-SA-4.0
% Author: Matthieu Perrin
% Part: 
% Section: 
% Sub-section: 
% Frame: 

\begingroup

\begin{frame}[fragile]{Machine à états d'un programme}

  Si l'exécution est séquentiellement cohérente, l'ensemble des exécutions est défini par une machine à états non-déterministe.

  \begin{block}{États du système}
    À chaque instant, la \alert{configuration globale} est définie par :
    \begin{itemize}
    \item Un ensemble de threads
    \item L'état local de chaque thread
      \begin{itemize}
      \item valeur de ses variables locales
      \item prochaine instruction à exécuter
      \item peut-il exécuter sa prochaine instruction ? 
      \end{itemize}
    \item L'état de la mémoire
      \begin{itemize}
      \item ensemble de variables partagées
      \item valeur de chacune d'entre elles
      \end{itemize}
    \end{itemize}
  \end{block}

  \begin{block}{Transitions}
    Un \alert{ordonnanceur} décide quel thread exécutera sa prochaine instruction.
  \end{block}
\end{frame}

\endgroup
\endinput
