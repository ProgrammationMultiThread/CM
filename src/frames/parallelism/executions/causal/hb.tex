% SPDX-License-Identifier: CC-BY-SA-4.0
% Author: Matthieu Perrin
% Part: 
% Section: 
% Sub-section: 
% Frame: 

\begingroup

\begin{frame}{La relation ``happened before'' (ordre causal)}

  \begin{shadequote}{Leslie Lamport}
    The \alert{concept of time} is fundamental to our way of
    thinking. \alert{It is derived} from the more basic concept of
    the \alert{order in which events occur}. (...)

    However, we will see that this concept must be carefully reexamined when considering events in a distributed system. (...)

    In a distributed system, it is sometimes impossible to
    say that one of two events occurred first. \alert{The relation
      "happened before" is therefore only a partial ordering
      of the events in the system.}
  \end{shadequote}

  \begin{block}{Définitions}
    \begin{itemize}
    \item $e\mapsto e'$ si l'une des trois conditions s'applique :    
      \begin{itemize}
      \item \structure{Ordre de programme :} le même thread produit $e$ puis $e'$
      \item \structure{Précédence sémantique :} (précisé plus tard)
        \begin{itemize}
        \item Par exemple, $e = t.start()$ et $e'$ produit par $t$
        \item Par exemple, $e$ produit par $t$ et $e' = t.join()$
        \end{itemize}
      \item \structure{Transitivité :} $\exists e'', e\mapsto e'' \mapsto e'$
      \end{itemize}
    \item $e$ et $e'$ sont \structure{concurrents} $(e || e')$ si $e\not\mapsto e'$ et $e'\not\mapsto e$.
    \end{itemize}
  \end{block}
  \alert{Attention :} $\mapsto$ est seulement un ordre \alert{partiel} !

  \footnoteref{L. Lamport. \textit{Time, clocks, and the ordering of events in a distributed system.} CACM (1978)}

\end{frame}

\endgroup
\endinput
