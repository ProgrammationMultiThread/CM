% SPDX-License-Identifier: CC-BY-SA-4.0
% Author: Matthieu Perrin
% Part: 
% Section: 
% Sub-section: 
% Frame: 

\begingroup

\begin{frame}{Caractérisation du progrès global}

  \begin{block}{Deux définitions équivalentes si le nombre de threads est fini}
    Un objet $O$ vérifie le progrès global si, pour toute exécution :
    \begin{enumerate}
    \item Pour tout suffixe de l'exécution, il existe un thread dont toutes les opérations sur $O$ se terminent.
    \item Si l'exécution contient un nombre fini d'opérations sur $O$, elles se terminent toutes.
    \end{enumerate}
  \end{block}

  \begin{exampleblock}{Idée de démonstration : $1 \implies 2$}
    Par récurrence sur le nombre de threads dans l'exécution :
    \begin{enumerate}
    \item S'il n'y a qu'un seul thread, toutes ses opérations se terminent
    \item S'il y a $n+1$ threads, l'un d'entre eux termine toutes ses opérations,
      et on se retrouve avec un suffixe avec $n$ threads. 
    \end{enumerate}
  \end{exampleblock}

  \begin{exampleblock}{Idée de démonstration : $2 \implies 1$}
    Par l'absurde : si aucun thread ne terminait l'une de ses opérations, il y aurait un nombre fini d'opérations dans l'exécution, donc elles termineraient toutes. 
  \end{exampleblock}

\end{frame}

\endgroup
\endinput
