% SPDX-License-Identifier: CC-BY-SA-4.0
% Author: Matthieu Perrin
% Part: 
% Section: 
% Sub-section: 
% Frame: 

\begingroup

\begin{frame}
  \frametitle{Une solution pour éviter les deadlocks}
  \vFill
  \begin{alertblock}{Composition}
    Des deadlocks peuvent se produire quand plusieurs verrous sont utilisés.
  \end{alertblock}
  \vFill
  \begin{block}{Théorème}
    \begin{itemize}
    \item Supposons qu'il y a un ordre sur les verrous $v_1 \le v_2 \le \dots \le v_n$.
    \item Supposons qu'aucun thread ne cherche à obtenir $v$ alors qu'il possède déjà $v'$ avec $v\le v'$.
    \item Alors l'exécution ne comporte pas de deadlock.
    \end{itemize}
  \end{block}

  \vFill
  \begin{alertblock}{En TD}
    Utilisez le théorème ci-dessus pour résoudre le problème du dîner des philosophes
  \end{alertblock}
  \vFill
\end{frame}

\endgroup
\endinput
