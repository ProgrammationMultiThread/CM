% SPDX-License-Identifier: CC-BY-SA-4.0
% Author: Matthieu Perrin
% Part: 
% Section: 
% Sub-section: 
% Frame: 

\begingroup

\begin{frame}[fragile]{Le moniteur universel en Java (\lstinline|java.lang.Object|)}

  \vfill    
  \begin{block}{\lstinline|public final void java.lang.Object.wait() throws InterruptedException|}
    \begin{itemize}
    \item Met le thread en attente sur l'objet verrouillé.
    \item Il doit \alert{posséder} le verrou de l'objet sur lequel il se met en attente.
    \item Lorsqu'il entre en attente, il \alert{relâche}  le verrou.
    \end{itemize}
  \end{block}

  \vfill    
  \begin{block}{\lstinline|public final void java.lang.Object.notify()|}
    \begin{itemize}
    \item Réveille un des threads en attente sur l'objet verrouillé.
    \item Le thread réveillé doit \alert{acquérir} le verrou.
    \end{itemize}
  \end{block}

  \vfill    
  \begin{block}{\lstinline|public final void java.lang.Object.notifyAll()|}
    \begin{itemize}
    \item Pareil mais réveille tous les threads en attente sur l'objet verrouillé.
    \end{itemize}
  \end{block}
  \vfill
  \begin{alertblock}{Point vocabulaire}
    \begin{itemize}
    \item Un thread demandant l'accès à un code déjà verrouillé est \alert{bloqué}.
    \item Un  thread qui  est mis  en pause  par l'opération wait est \alert{en attente}.      
    \end{itemize}
  \end{alertblock}
  \vfill
\end{frame}

\endgroup
\endinput
