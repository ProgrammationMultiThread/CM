% SPDX-License-Identifier: CC-BY-SA-4.0
% Author: Matthieu Perrin
% Part: 
% Section: 
% Sub-section: 
% Frame: 

\begingroup

\begin{frame}{Mot-clé \alert{synchronized} en Java}

  \begin{alertblock}{Moniteur d'exécution en Java}
    Un \alert{verrou} est associé à chaque :
    \begin{itemize}
    \item objet instancié dans la JVM,
    \item classe.
    \end{itemize}
  \end{alertblock}
  
  \begin{block}{Mot-clé \alert{synchronized}}
    \begin{itemize}
    \item Méthode  déclarée \alert{synchronized} :\\le  verrou est pris
      pour la durée de l'exécution de la méthode. 
    \item Bloc \alert{synchronized\{\}} :\\le  verrou est pris
      pour la durée de l'exécution du bloc.
    \end{itemize}
  \end{block}

  \pause

  \begin{minipage}{.47\textwidth}
    \begin{exampleblock}{Avantages}
      \begin{itemize}
      \item Simple d'utilisation.
      \item Mise en \oe uvre efficace.
      \end{itemize}
    \end{exampleblock}
  \end{minipage}
  \hfill
  \begin{minipage}{.47\textwidth}
    \begin{alertblock}{Inconvénients}
      \begin{itemize}
      \item Peu flexible dans l'utilisation.
      \item Une seule sorte de verrou 
      \end{itemize}
    \end{alertblock}
  \end{minipage}
\end{frame}

\endgroup
\endinput
