% SPDX-License-Identifier: CC-BY-SA-4.0
% Author: Matthieu Perrin
% Part: 
% Section: 
% Sub-section: 
% Frame: 

\begingroup

\tikzset{
  sshape/.style={
    text width=15mm,
    rectangle,
    rounded corners,
    fill=black!10,
  },
}

\newcommand\scontent[4]{\begin{tabular}{@{}r@{}c@{~}l@{}}
    \tiny \texttt{t1}     & \tiny : & \tiny \texttt{#1} \\[-.8mm]
    \tiny \texttt{t2}     & \tiny : & \tiny \texttt{#2} \\[-.8mm]
    \tiny \texttt{result} & \tiny = & \tiny \texttt{#3} \\[-.8mm]
    \tiny \texttt{lock}   & \tiny = & \tiny \texttt{#4}
\end{tabular}}

\begin{frame}{Incrémentations concurrentes}

%  \onBlock<2->[width=.65\textwidth]{Algorithme bloquant}{
% 
%    \vspace{-2mm}
%    \begin{itemize}
%    \item Point de linéarisation : n'importe où en SC
%    \end{itemize}
%    \centering
% 
%    \begin{tikzpicture}[y=8mm]\scriptsize
%      \draw[process, structure]    (0,2    ) node[left] (cpt)  {\texttt{cpt}}   +(.2 ,.15) node{$0$} +(0,0) -- (6.5,2   );
%      \draw[process, exampleColor] (0,1    ) node[left] (t1)   {\texttt{t1}}                                -- (6.5,1   );
%      \draw[process, alertColor]   (0,0    ) node[left] (t2)   {\texttt{t2}}                                -- (6.5,0   );
%      \draw[process, structure]    (0,-1   ) node[left] (val)  {\texttt{value}} +(.15,.12) node{$0$} +(0,0) -- (6.5,-1  );
%      \draw[process, structure]    (0,-1.5 ) node[left] (lock) {\texttt{lock}}                              -- (6.5,-1.5);
%      
%      \node[structure, operation, text width=26mm] (o11) at (2.4,1)    {$lock()$};
%      \node[structure, operation, right          ] (o21) at (o11.east) {$get()$};
%      \node[structure, operation, right          ] (o31) at (o21.east) {$set(2)$};
%      \node[structure, operation, right          ] (o41) at (o31.east) {$unlock()$};
%      \node[structure, operation                 ] (o10) at (.6,0)     {$lock()$};
%      \node[structure, operation, right          ] (o20) at (o10.east) {$get()$};
%      \node[structure, operation, right          ] (o30) at (o20.east) {$set(1)$};
%      \node[structure, operation, right          ] (o40) at (o30.east) {$unlock()$};
% 
%      \coordinate (lp11) at ($(o11.south) + (1.2,0)$);
% 
%      \begin{scope}[background]
%        \node[example, operation, fit=(o11)(o41), inner sep=.5mm] (o1) {};
%        \node[alert,   operation, fit=(o10)(o40), inner sep=.5mm] (o0) {};
%        \draw[alertColor,   ultra thick] (o10  |- lock) -- (o40 |- lock) ; 
%        \draw[exampleColor, ultra thick] (lp11 |- lock) -- (o41 |- lock) ; 
%      \end{scope}
%      
%      \draw[structure,    lin point] (lp11)      -- (lp11 |- lock) ; 
%      \draw[structure,    lin point] (o41.south) -- (o41 |- lock) ; 
%      \draw[structure,    lin point] (o10.south) -- (o10 |- lock) ; 
%      \draw[structure,    lin point] (o40.south) -- (o40 |- lock) ; 
%      \draw[structure,    lin point] (o20.south) -- (o20 |- val) node[above right]{$0$}; 
%      \draw[structure,    lin point] (o30.south) -- (o30 |- val) node[above right]{$1$}; 
%      \draw[structure,    lin point] (o21.south) -- (o21 |- val) node[above right]{$1$}; 
%      \draw[structure,    lin point] (o31.south) -- (o31 |- val) node[above right]{$2$}; 
%      \draw[exampleColor, lin point] (o1.north) -- (o1 |- cpt) node[above left]{$increment()$} node[above right]{$2$}; 
%      \draw[alertColor,   lin point] (o0.north) -- (o0 |- cpt) node[above left]{$increment()$} node[above right]{$1$}; 
%      
%    \end{tikzpicture}
%  }
  
  \on[y=-3mm]{
    \begin{tikzpicture}[automaton, x=24mm, y=16mm]

      \state[sshape,initial]                       (04) at (0,4) {\scontent {\example{lock()}}   {\alert{lock()}}   {0} {free} };
      \state[sshape]                               (03) at (0,3) {\scontent {lock()}             {\alert{get()}}    {0} {\alert{t2}} };
      \state[sshape]                               (02) at (0,2) {\scontent {lock()}             {\alert{set(1)}}   {0} {\alert{t2}} };
      \state[sshape]                               (01) at (0,1) {\scontent {lock()}             {\alert{unlock()}} {1} {\alert{t2}} };
      \state[sshape]                               (00) at (0,0) {\scontent {\example{lock()}}   {\textsc{term.}}   {1} {free} };
      
      \state[sshape]                               (14) at (1,4) {\scontent {\example{get()}}    {lock()}           {0} {\example{t1}} };
      \state[sshape]                               (24) at (2,4) {\scontent {\example{set(1)}}   {lock()}           {0} {\example{t1}} };
      \state[sshape]                               (34) at (3,4) {\scontent {\example{unlock()}} {lock()}           {1} {\example{t1}} };
      
      \state[sshape]                               (10) at (1,0) {\scontent {\example{get()}}    {\textsc{term.}}   {1} {\example{t1}} };
      \state[sshape]                               (20) at (2,0) {\scontent {\example{set(2)}}   {\textsc{term.}}   {1} {\example{t1}} };
      \state[sshape]                               (30) at (3,0) {\scontent {\example{unlock()}} {\textsc{term.}}   {2} {\example{t1}} };
      
      \state[sshape]                               (44) at (4,4) {\scontent {\textsc{term.}}     {\alert{lock()}}   {1} {free} };
      \state[sshape]                               (43) at (4,3) {\scontent {\textsc{term.}}     {\alert{get()}}    {1} {\alert{t2}} };
      \state[sshape]                               (42) at (4,2) {\scontent {\textsc{term.}}     {\alert{set(2)}}   {1} {\alert{t2}} };
      \state[sshape]                               (41) at (4,1) {\scontent {\textsc{term.}}     {\alert{unlock()}} {2} {\alert{t2}} };
      \state[sshape, accepting, fill=structure!20] (40) at (4,0) {\scontent {\textsc{term.}}     {\textsc{term.}}   {2} {free} };

      \path[alert]   (04) edge (03);
      \path[alert]   (03) edge (02);
      \path[alert]   (02) edge (01);
      \path[alert]   (01) edge (00);

      \path[alert]   (44) edge (43);
      \path[alert]   (43) edge (42);
      \path[alert]   (42) edge (41);
      \path[alert]   (41) edge (40);

      \path[example] (04) edge (14);
      \path[example] (14) edge (24);
      \path[example] (24) edge (34);
      \path[example] (34) edge (44);

      \path[example] (00) edge (10);
      \path[example] (10) edge (20);
      \path[example] (20) edge (30);
      \path[example] (30) edge (40);

    \end{tikzpicture}
  }
  
\end{frame}

\endgroup
\endinput
