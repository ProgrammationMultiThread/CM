% SPDX-License-Identifier: CC-BY-SA-4.0
% Author: Matthieu Perrin
% Part: 
% Section: 
% Sub-section: 
% Frame: 

\begingroup

\begin{frame}[fragile]{Exemple de programme parallèle}

  \begin{block}{Calcul de $\pi$ par la méthode de Monte-Carlo}
    \centering
    \begin{tikzpicture}[x=2cm, y=2cm]
      \draw[-latex] (0,-0.25) -- (0,1.25);
      \draw[-latex] (-0.25,0) -- (1.25,0);
      \draw[fill=alert!30]   (0,0) rectangle (1,1);
      \draw[fill=example!30] (0,0) -- (1,0) arc (0:90:1) -- (0,0);
      \node[alert,  anchor=north east] at (1,1) {$\mathcal{A_\mathcal{C}}$};
      \node[example,anchor=south west] at (0,0) {$\mathcal{A_\mathcal{D}}$};
      \node[anchor=east]               at (0,1) {1};
      \node[anchor=north]              at (1,0) {1};
      \node[anchor=north east]         at (0,0) {0};
      
      \node[anchor=west] at (1.5,.5) {$
        \begin{array}{rcl}
          \mathcal{A_\mathcal{C}} &=& 1             \\[2mm]
          \mathcal{A_\mathcal{D}} &=& \frac{\pi}{4} \\[2mm]
          \pi &=& 4\times \mathbb{P}\left(x^2 + y^2 < 1 \mid \langle x, y\rangle \in [0; 1]^2\right)
        \end{array}
        $};
    \end{tikzpicture}
  \end{block}

  \begin{exampleblock}{Expérience}
    \begin{itemize}
    \item Tirer \lstinline{nbIterations} points dans l'intervalle $[0; 1]^2$
    \item Paralléliser entre \lstinline{nbThreads} threads
    \item Mesurer le temps d'exécution en fonction de \lstinline{nbIterations} et \lstinline{nbThreads}
    \end{itemize}
  \end{exampleblock}

  \footnoterefjava{Comparer \lstinline{cm3/Pi1_Sequential.java} et \lstinline{cm3/Pi2_Parallel.java}}

\end{frame}

\endgroup
\endinput
