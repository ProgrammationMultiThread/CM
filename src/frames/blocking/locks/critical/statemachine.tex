% SPDX-License-Identifier: CC-BY-SA-4.0
% Author: Matthieu Perrin
% Part: 
% Section: 
% Sub-section: 
% Frame: 

\begingroup

\tikzset{
  sshape/.style={
    text width=15mm,
    rectangle,
    rounded corners,
    fill=black!10,
  },
}

\newcommand\scontent[3]{\begin{tabular}{@{}r@{}c@{~}l@{}}
    \tiny \texttt{t1}     & \tiny : & \tiny \texttt{#1} \\[-.8mm]
    \tiny \texttt{t2}     & \tiny : & \tiny \texttt{#2} \\[-.8mm]
    \tiny \texttt{result} & \tiny = & \tiny \texttt{#3} 
\end{tabular}}

\begin{frame}{Ce que l'on a}
  \centering

  \begin{tikzpicture}[automaton, x=25mm, y=12mm]

    \state[sshape,initial]                       (03) at (0,3) {\scontent {\example{get()}}  {\alert{get()}}  {0} };
    \state[sshape]                               (02) at (0,2) {\scontent {\example{get()}}  {\alert{put(1)}} {0} };
    \state[sshape]                               (00) at (0,0) {\scontent {\example{get()}}  {\textsc{term.}} {1} };
    
    \state[sshape]                               (13) at (1,3) {\scontent {\example{put(1)}} {\alert{get()}}  {0} };
    \state[sshape]                               (12) at (1,2) {\scontent {\example{put(1)}} {\alert{put(1)}} {0} };
    \state[sshape]                               (11) at (1,1) {\scontent {\example{put(1)}} {\textsc{term.}} {1} };
    \state[sshape]                               (10) at (1,0) {\scontent {\example{put(2)}} {\textsc{term.}} {1} };
    
    \state[sshape]                               (22) at (2,2) {\scontent {\textsc{term.}}   {\alert{put(1)}} {1} };
    \state[sshape, accepting, fill=alert!20]     (21) at (2,1) {\scontent {\textsc{term.}}   {\textsc{term.}} {1} };
    
    \state[sshape]                               (33) at (3,3) {\scontent {\textsc{term.}}   {\alert{get()}}  {1} };
    \state[sshape]                               (32) at (3,2) {\scontent {\textsc{term.}}   {\alert{put(2)}} {1} };
    \state[sshape, accepting, fill=structure!20] (30) at (3,0) {\scontent {\textsc{term.}}   {\textsc{term.}} {2} };

    \path[alert]   (03) edge (02);
    \path[alert]   (02) edge (00);
    \path[alert]   (13) edge (12);
    \path[alert]   (12) edge (11);
    \path[alert]   (22) edge (21);
    \path[alert]   (33) edge (32);
    \path[alert]   (32) edge (30);

    \path[example] (03) edge (13);
    \path[example] (13) edge (33);
    \path[example] (02) edge (12);
    \path[example] (12) edge (22);
    \path[example] (11) edge (21);
    \path[example] (00) edge (10);
    \path[example] (10) edge (30);
  \end{tikzpicture}

  \begin{block}{Quelques entrelacements possibles}
    \begin{itemize}
    \item ${\color{exampleColor}t_1:get();~} {\color{exampleColor}t_1:put(1);~} {\color{alertColor}t_2:get();~} {\color{alertColor}t_2:put(2);}$ \hspace{5mm} \structure{$result=2$}
    \item ${\color{exampleColor}t_1:get();~} {\color{alertColor}t_2:get();~} {\color{exampleColor}t_1:put(1);~} {\color{alertColor}t_2:put(1);}$ \hspace{5mm} \alert{$result=1$}
    \end{itemize}
  \end{block}

\end{frame}

\endgroup
\endinput
