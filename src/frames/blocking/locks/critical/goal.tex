% SPDX-License-Identifier: CC-BY-SA-4.0
% Author: Matthieu Perrin
% Part: 
% Section: 
% Sub-section: 
% Frame: 

\begingroup

\tikzset{
  sshape/.style={
    text width=20mm,
    rectangle,
    rounded corners,
    fill=black!10,
  },
}

\newcommand\scontent[3]{\begin{tabular}{@{}r@{}c@{~}l@{}}
    \tiny \texttt{t1}     & \tiny : & \tiny \texttt{#1} \\[-.8mm]
    \tiny \texttt{t2}     & \tiny : & \tiny \texttt{#2} \\[-.8mm]
    \tiny \texttt{result} & \tiny = & \tiny \texttt{#3} \\[-.8mm]
\end{tabular}}

\begin{frame}[fragile]{Ce que l'on voudrait}

  \begin{block}{Rappel : exécution séquentiellement cohérente}
    Le résultat observable \alert<2>{d'une exécution} est le même que celui d'un entrelacement de l'ordre ``happened before''
  \end{block}

  \vspace{3mm}
  \begin{lstlisting}
    interface Counter {
      void increment() ;
      int get() ;
    }
  \end{lstlisting}
  \vspace{3mm}

  \on[y=5mm, x=.2\textwidth]{
    \begin{tikzpicture}[automaton, x=30mm, y=12mm]
      \state[sshape,initial]                       (01) at (0,1) {\scontent {\example{increment()}} {\alert{increment()}}  {0} };
      \state[sshape]                               (00) at (0,0) {\scontent {\example{increment()}} {\textsc{term.}}       {1} };
      \state[sshape]                               (11) at (1,1) {\scontent {\textsc{term.}}        {\alert{increment()}}  {1} };
      \state[sshape, accepting, fill=structure!20] (10) at (1,0) {\scontent {\textsc{term.}}        {\textsc{term.}}       {2} };

      \path[alert]   (01) edge (00);
      \path[alert]   (11) edge (10);
      \path[example] (01) edge (11);
      \path[example] (00) edge (10);
    \end{tikzpicture}
  }
  
  \begin{block}{Entrelacements possibles}
    \vspace{-2mm}
    \begin{itemize}
    \item ${\color{exampleColor}t_1:increment();~} {\color{alertColor}t_2:increment();~}$ \hspace{5mm} \structure{$result=2$}
    \item ${\color{alertColor}t_2:increment();~} {\color{exampleColor}t_1:increment();~}$ \hspace{5mm} \structure{$result=2$}
    \end{itemize}
  \end{block}

  \pause
  \vspace{-1mm}

  \begin{block}{Objet partagé séquentiellement cohérent}
    Le résultat observable \alert{des opérations sur l'objet partagé}
    d'une exécution est le même que celui d'un entrelacement de l'ordre ``happened before''
  \end{block}

  
%  \begin{block}{Notion d'atomicité}
%    \vspace{-2mm}
%    \begin{itemize}
%    \item À quelle granularité définit-on les \structure{événements} d'une exécution ?  
%    \item \alert{Section critique} : portion de code qui doit être exécutée \structure{atomiquement}
%    \end{itemize}
%  \end{block}

\end{frame}

\endgroup
\endinput
