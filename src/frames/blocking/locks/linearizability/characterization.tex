% SPDX-License-Identifier: CC-BY-SA-4.0
% Author: Matthieu Perrin
% Part: 
% Section: 
% Sub-section: 
% Frame: 

\begingroup

\begin{frame}{Caractérisation}

  \begin{block}{Point de linéarisation}
    Un objet partagé est linéarisable si, et seulement si : \\
    le résultat observable des opérations sur l'objet partagé d'une exécution est le même que si
    \alert{chaque opération avait lieu en un point atomique
      situé entre son début et sa fin.}
  \end{block}

  \vfill
  \begin{exampleblock}{Exemple}

    \begin{center}
      \scalebox{.7}{
        \begin{tikzpicture}
          \draw[exampleColor, -latex] (0,4) node[left]{$counter$} -- (10,4);
          \draw[-latex] (0,3) node[left]{$T_1$} -- (10,3);
          \draw[-latex] (0,2) node[left]{$T_2$} -- (10,2);

          \draw[exampleColor, thick, densely dotted] (3  , 2) -- (3  , 4) node{$\bullet$};
          \draw[exampleColor, thick, densely dotted] (7.5, 3) -- (7.5, 4) node{$\bullet$};

          \draw[exampleColor, fill=exampleColor!20, rounded corners] (1 , 1.6) rectangle (5 , 2.4);
          \draw[exampleColor, fill=exampleColor!20, rounded corners] (6, 2.6) rectangle (9, 3.4);

          \draw[exampleColor] (3,2) node{$increment()$ };
          \draw[exampleColor] (7.5,3) node{$get()$};
        \end{tikzpicture}
      }
    \end{center}
  \end{exampleblock}

  \vfill
  \small
  \begin{alertblock}{Propriétés}
    \vspace{-2mm}
    \begin{itemize}
    \item La linéarisabilité est composable
    \item La linéarisabilité implique la cohérence séquentielle
    \item Si chaque objet est linéarisable, l'exécution est séquentiellement cohérente
    \end{itemize}
  \end{alertblock}
\end{frame}

\endgroup
\endinput
