% SPDX-License-Identifier: CC-BY-SA-4.0
% Author: Matthieu Perrin
% Part: 
% Section: 
% Sub-section: 
% Frame: 

\begingroup

\begin{frame}[fragile]{Modèle de mémoire en Java}
  \begin{block}{Le mot-clé \lstinline{volatile} en Java}
    \begin{itemize}
    \item Les variables \lstinline{volatile} sont linéarisables
    \item Gestion par le système
      \begin{itemize}
      \item Limite les optimisations du compilateur
      \item Limite l'utilisation des caches
      \end{itemize}
    \end{itemize}
  \end{block}
  \vfill
  \begin{alertblock}{Toute variable doit être :}
    \begin{itemize}
    \item soit déclarée \lstinline{final}
    \item soit déclarée \lstinline{volatile}
    \item soit uniquement accédée séquentiellement
      \begin{itemize}
      \item exemple des variables locales
      \item par exemple en section critique
      \item lectures et écritures ordonnées par ``happened before''
      \end{itemize}
    \end{itemize}
  \end{alertblock}
  \vfill
  \begin{citing}
  \item \href{https://docs.oracle.com/javase/specs/jls/se7/html/jls-17.html#jls-17.4}{Voir la spécification officielle de la JVM, chapitre 17.4}
  \end{citing}
\end{frame}

\endgroup
\endinput
