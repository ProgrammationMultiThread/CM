% SPDX-License-Identifier: CC-BY-SA-4.0
% Author: Matthieu Perrin
% Part: 
% Section: 
% Sub-section: 
% Frame: 

\begingroup

\begin{frame}[fragile]{Retour sur le calcul de $\pi$}

  \vspace{-10mm}
  \begin{lstlisting}
    public class PiMonteCarlo implements Runnable {
      private int result = 0;
      public void run() {
        for(int i = 0; i<nbIterations; i++) {
          double x = random.nextDouble();
          double y = random.nextDouble();
          if(x*x + y*y < 1) (*\Structure{result++;}*)
        }
      }
      static int experiment() {
        for(int i = 0; i<nbThreads; i++) {
          tasks[i] = new PiMonteCarlo();
          threads[i] = new Thread(tasks[i]);
        }
        for(var t : threads) t.start();
        for(var t : threads) (*\Alert{t.join();}*)
        int result = 0;
        for(var t : tasks) (*\Structure{result+=t.result;}*)
        return result;
      }
    }
  \end{lstlisting}

  \onBlock[bottom]{Happens Before}{
    \centering
    \begin{tikzpicture}[x=40mm, y=15mm, anchor=mid]
      \footnotesize
      \node[operation, fill=black!10] (o1) at (1,0) {\lstinline{result++}};
      \node[operation, fill=black!10] (o2) at (2,0) {\lstinline{t.join()}};
      \node[operation, fill=black!10] (o3) at (3,0) {\lstinline{t.result}};

      \path[-latex, structure] (o1) edge node[above]{ordre de} node[below]{programme}       (o2);
      \path[-latex, alert]     (o2) edge node[above]{ordre de} node[below]{synchronisation} (o3);
    \end{tikzpicture}
  }

\end{frame}

\endgroup
\endinput
