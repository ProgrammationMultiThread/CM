% SPDX-License-Identifier: CC-BY-SA-4.0
% Author: Matthieu Perrin
% Part: 
% Section: 
% Sub-section: 
% Frame: 

\begingroup

\begin{frame}[fragile]{Limite de \lstinline{volatile} : les tableaux}
  \begin{exampleblock}{Tableaux volatiles}
    \begin{lstlisting}[numbers=none]
      volatile int array[];
      array = new int[10];          // volatile
      array[5] = 3;                 // pas volatile
    \end{lstlisting}    
  \end{exampleblock}
  \pause
  \begin{alertblock}{Première solution}
    \begin{lstlisting}[numbers=none]
      class VolatileInt{public volatile int x=0;}
      VolatileInt array[] = new VolatileInt[10];
      array[5] = new VolatileInt(); // pas volatile
      array[5].x = 3;               // volatile
    \end{lstlisting}
  \end{alertblock}
  \pause
  \begin{block}{Meilleure solution: \lstinline{java.util.concurrent.atomic.AtomicIntegerArray}}
    \begin{lstlisting}[numbers=none]
      AtomicIntegerArray array = new AtomicIntegerArray(10);
      array.set(5, 3);              // volatile
    \end{lstlisting}
  \end{block}    
\end{frame}

\endgroup
\endinput
