% SPDX-License-Identifier: CC-BY-SA-4.0
% Author: Matthieu Perrin
% Part: 
% Section: 
% Sub-section: 
% Frame: 

\begingroup

\begin{frame}{Linéarisabilité}
  \begin{block}{Définition}
    Un objet partagé est linéarisable si : \\
    le résultat observable des opérations sur l'objet partagé d'une exécution est le même que celui d'un entrelacement de l'ordre ``happened before''
    \alert{tel que, si une opération $e$ se termine avant qu'une autre opération $e'$ ne commence, alors $e$ doit précéder $e'$ dans l'entrelacement}
  \end{block}

  \vfill
  \begin{exampleblock}{Exemple}
    \vspace{-3mm}
    \begin{center}
      \scalebox{.7}{
        \begin{tikzpicture}
          \draw[-latex] (0,3) node[left]{$T_1$} -- (10,3);
          \draw[-latex] (0,2) node[left]{$T_2$} -- (10,2);

          \draw[exampleColor, fill=exampleColor!20, rounded corners] (1 , 1.6) rectangle (5 , 2.4);
          \draw[exampleColor, fill=exampleColor!20, rounded corners] (6, 2.6) rectangle (9, 3.4);

          \draw[exampleColor] (3,2) node{$increment()$ };
          \draw[exampleColor] (7.5,3) node{$get()$};
        \end{tikzpicture}
      }
    \end{center}

    Un seul ordre de linéarisation possible : 
    \begin{itemize}
    \item $increment() ;~ get()$
      \begin{itemize}
      \item La lecture doit retourner 1
      \end{itemize}
    \end{itemize}
  \end{exampleblock}

\end{frame}

\endgroup
\endinput
