% SPDX-License-Identifier: CC-BY-SA-4.0
% Author: Matthieu Perrin
% Part: 
% Section: 
% Sub-section: 
% Frame: 

\begingroup

\begin{frame}{Modèle de Mémoire du langage Java (JMM)}

  \vspace{-3mm}

  \structure{Rappel :} Si l'exécution est \structure{séquentiellement cohérente} (SC),
  l'ensemble des exécutions est décrit par une sémantique d'entrelacement

  \begin{alertblock}{Problème}
    \begin{itemize}
    \item Les programmes Java \alert{ne sont pas} SC par défaut
    \item Comment raisonner sur un programme concurrent ?
    \item<2-> Un programme est \structure{correctement synchronisé} si aucune de ses exécutions SC ne contient de \alert{data race}
    \item<2-> Tout programme correctement synchronisé est SC
    \end{itemize}
  \end{alertblock}

  \vspace{2mm}
  \pause
  
  \begin{shadequote}{\href{https://docs.oracle.com/javase/specs/jls/se7/html/jls-17.html\#jls-17.4}{Spécification du langage Java (JLS §17.4)}}
    When a program contains two conflicting accesses that are not ordered by a \alert<3>{happens-before} relationship, it is said to contain a \alert{data race}.
  \end{shadequote}

  \vspace{2mm}
  \pause
  
  \structure{Rappel :}  la relation \alert{happens-before} est un ordre partiel formé de :
  \begin{itemize}
  \item un \structure{ordre de programme} fait le lien avec la sémantique séquentielle
  \item une \structure{relation synchronises-with} fait le lien entre les threads
  \end{itemize}
  
\end{frame}

\endgroup
\endinput
