% SPDX-License-Identifier: CC-BY-SA-4.0
% Author: Matthieu Perrin
% Part: 
% Section: 
% Sub-section: 
% Frame: 

\begingroup

\begin{frame}[fragile]{Comment éviter les data races ?}

  \begin{block}{Règle d'or pour le langage Java}
    Pour chaque variable $x$, il faut s'assurer que $x$ est dans l'un des cas suivants : 
    \begin{itemize}
    \item $x$ est déclarée \lstinline{final}
      \begin{itemize}
      \item \structure{Adoptez un style de programmation fonctionnelle}
      \end{itemize}
    \item $x$ n'est accessible que par un seul thread
      \begin{itemize}
      \item \structure{Si $x$ est un membre d'une classe $C$, s'applique à toutes les instances de $C$}
      \end{itemize}
    \item $x$ est déclarée \lstinline{volatile}
      \begin{itemize}
      \item \structure{Utile pour la communication simple entre threads}
      \end{itemize}
    \item $x$ est \alert{toujours} accédée dans un bloc \lstinline{synchronized} sur le même moniteur
      \begin{itemize}
      \item \structure{Mais attention à ne pas trop limiter le parallélisme}
      \end{itemize}
    \item $x$ n'est pas sujet à une data race
      \begin{itemize}
      \item \structure{Prouvez-le en commentaire}
      \end{itemize}
    \end{itemize}
  \end{block}

\end{frame}

\endgroup
\endinput
