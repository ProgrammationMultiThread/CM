% SPDX-License-Identifier: CC-BY-SA-4.0
% Author: Matthieu Perrin
% Part: 
% Section: 
% Sub-section: 
% Frame: 

\begingroup

\begin{frame}[fragile]{Modèle de mémoire en C++}

  \vfill

  \begin{block}{Le mot-clé \lstinline{volatile} en C++}
    \begin{itemize}
    \item Limite certaines optimisations locales du compilateur
    \item \alert{Ne gère pas la synchronisation entre threads}
    \end{itemize}
  \end{block}

  \vfill

  \begin{block}{Barrières de mémoire en C++}
    \begin{itemize}
    \item \lstinline{void std::atomic_thread_fence( std::memory_order order )}
    \item Choix des arcs ajoutés à l'ordre $\sw$ : read/write, write/read...
    \end{itemize}
    \vspace{1mm}
    \begin{shadequote}{Phil Karlton}
      There are only two hard things in Computer Science: \\cache invalidation and naming things.
    \end{shadequote}
  \end{block}

  \vspace{-3mm}
  
  \begin{alertblock}{C++ Core Guidelines}
    \begin{itemize}
    \item \href{https://isocpp.github.io/CppCoreGuidelines/CppCoreGuidelines#Rconc-volatile}{\alert{CP.8 :} Don’t try to use \lstinline{volatile} for synchronization}
    \item Utiliser \lstinline{template <class T> struct std::atomic}
    \end{itemize}
  \end{alertblock}

  \vspace{-1mm}

  \begin{citing}
  \item Tutoriel de Ori Lahav sur les modèles de mémoire en C++
  \end{citing}

\end{frame}

\endgroup
\endinput
