% SPDX-License-Identifier: CC-BY-SA-4.0
% Author: Matthieu Perrin
% Part: 
% Section: 
% Sub-section: 
% Frame: 

\begingroup

\tikzset{
  sshape/.style={
    text width=20mm,
    rectangle,
    rounded corners,
    fill=black!10,
  },
}

\newcommand\scontent[3]{\begin{tabular}{@{}r@{~}c@{~}l@{}}
    \tiny \structure{\texttt{taken}}   & \tiny \structure{=}  & \tiny \structure{#1}   \\[-1mm]
    \tiny \color{black} $t_1$ & \tiny \color{black}: & \tiny \color{black}#2  \\[-1mm]
    \tiny \color{black} $t_2$ & \tiny \color{black}: & \tiny \color{black}#3
\end{tabular}}

\newcommand\thetaken{\texttt{taken}}
\newcommand\whiletaken{\texttt{while(taken)}}
\newcommand\takentrue{\texttt{taken = true}}
\newcommand\SC{\alert{section critique}}


\begin{frame}[fragile]{Un premier essai}

  \vspace{-30mm}
  \begin{lstlisting}[gobble=2]
    class NaiveLock1 implements Lock {
      private (*\Example{boolean taken = false;}*)
      public void lock() {
        while( (*\Structure{taken}*) );
        (*\Structure{taken = true;}*)
      }
      public void unlock() {
        (*\Alert{taken = false;}*)
      }
    }
  \end{lstlisting}
  
  \obExampleBlock<1>[anchor=north, y=5mm]{Exercice}{
    \vspace{5mm}
    Deux threads, $t_1$ et $t_2$, veulent entrer concurrament en section critique.
    \begin{itemize}
    \item Tracer la machine à états du système. 
    \item Montrer que l'\example{exclusion mutuelle} n'est pas respectée
    \item Montrer que la \example{starvation-freedom} n'est pas respectée
    \end{itemize}
  }

  \onExampleBlock<2->[anchor=north, y=5mm]{Exercice}{\centering
    \begin{tikzpicture}[automaton, x=10mm,y=12mm]
      \state[alert ob=<3-4>,example ob=<2>, sshape,initial] (s02) at (0,2) {\scontent{\False}{\whiletaken}{\whiletaken}};
      \state[alert ob=<3-4>,sshape,]                        (s12) at (3,2) {\scontent{\False}{\takentrue}{\whiletaken}};
      \state[alert ob=<4>,sshape,]                          (s22) at (6,2) {\scontent{\True} {\SC}          {\whiletaken}};
      \state[sshape,]                                       (s01) at (0,1) {\scontent{\False}{\whiletaken}{\takentrue}};
      \state[alert ob=<3>,sshape,]                          (s11) at (3,1) {\scontent{\False}{\takentrue}{\takentrue}};
      \state[sshape,]                                       (s21) at (6,1) {\scontent{\True} {\SC          }{\takentrue}};
      \state[sshape,]                                       (s00) at (0,0) {\scontent{\True} {\whiletaken}{\SC}          };
      \state[alert ob=<3>,sshape,]                          (s10) at (3,0) {\scontent{\True} {\takentrue}{\SC}          };
      \state[alert ob=<3>, sshape,accepting]                (s20) at (6,0) {\scontent{\True} {\SC}          {\SC}};

      \path[structure, alert ob=<3-4>] (s02) edge                          (s12);
      \path[structure, alert ob=<4>]   (s12) edge                          (s22);
      \path[structure]                 (s01) edge                          (s11);
      \path[structure]                 (s11) edge                          (s21);
      \path[structure]                 (s00) edge[loop right, looseness=2] (s00);
      \path[structure, alert ob=<3>]   (s10) edge                          (s20);
      \path[structure]                 (s02) edge                          (s01);
      \path[structure]                 (s01) edge                          (s00);
      \path[structure, alert ob=<3>]   (s12) edge                          (s11);
      \path[structure, alert ob=<3>]   (s11) edge                          (s10);
      \path[structure, alert ob=<4>]   (s22) edge[loop right, looseness=2] (s22);
      \path[structure]                 (s21) edge                          (s20);
      \path[alert, ob=<4>]             (s22) edge[bend right=5mm]          (s02);
    \end{tikzpicture}
  }

\end{frame}

\endgroup
\endinput
